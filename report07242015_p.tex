\begin{comment}
\documentclass[a4paper,12pt]{article}
\usepackage{comment}
\usepackage{import}
%citepath="/home/bruno/users/weh40/sims_phosim/data/SEDs/SEDs" # my default path
%dirlist=["BD_H2O_1", "BD_O3_1run", "BD_O2_1run", "H2O-new_4run-img", "H2O-aperture_1run", "O3_2run-100", "O2_1run" ]


\usepackage{xspace} 
\usepackage{graphicx}
\usepackage{subfig}
\usepackage{float}
\usepackage{placeins}
\usepackage[document]{ragged2e}
\usepackage{amsmath}
\usepackage{mathtools}

\DeclareGraphicsExtensions{.pdf,.png}
\graphicspath{{/home/bruno/users/weh40/Pictures/}}
\newcommand{\citp}{/home/bruno/users/weh40/sims_phosim/data/SEDs/SEDs/back/}
\newcommand{\fa}{BD_H2O_1/}
\newcommand{\fb}{BD_O3_1run/}
\newcommand{\fc}{BD_O2_1run/}
\newcommand{\fd}{H2O-new_4run-img/}
\newcommand{\fe}{H2O-aperture_1run/}
\newcommand{\ff}{O3_2run-100/}
\newcommand{\fg}{O2_1run/}
\newcommand{\R0}{Intcolors/tpxall_R0_order0/}
\let\oldtabular\tabular
\renewcommand{\tabular}{\footnotesize\oldtabular}
\usepackage{amsmath}
\begin{document}
\end{comment}


\section{Results}

\subsection{wDs}


With scipy.optimize.minimize, multiple dimensional fit can be done relatively fast and easily.

In practice, we will choose a power function, which leads to slightly improved error compared with the logarithmic function. 

Specifically, we prefer $y_{\rm power} = (bx_{\rm H2O}+1)^I -1$ over $y_{\rm log} = log_{10}(bx_{\rm H2O}+1)$ from experience. The power function here means the base contains the variable $x_{\rm H2O}$. (the $H_2O$ norm value, as defined in Sec. 1.1), y is the band magnitude shift, b and I are coefficients.

Because of the interdependence of multiple color terms, an approach is to perform a Principal Componnets Analysis (PCA), so as to seperate out the independent dimensions with the most variance. The package for pca is matplotlib.mlab.mlabPCA, which automatically normalizes the variance of each input dimension. In practice, we assumed that dimensions with much less variances in colors can be neglected.

Of 8 colors: u-g, g-r, g-i, g-z, g-y, r-i, r-z, r-y, the intrinsic variances are of order 1: [0.443, 0.337, 0.537, 0.661, 0.790, 0.200, 0.324, 0.454]

The weight coefficients (descaled from the output weight matrix) for first three eigenvectors  are:


W1: [-0.7965731 , -0.79756769, -0.7977423 , -0.79782595, -0.79783285, -0.79773173, -0.7977184 , -0.79736486] ,


W2: [ 2.50661256,  0.49760719,  0.01441818, -0.15567855, -0.31273392,  -0.79864561, -0.83446674, -0.9137465 ],


W3: [  0.6371916 , -0.98133088, -0.69430005, -0.38027843,  0.24019528,  -0.21105062,  0.2444841 ,  1.14636762] 



The fractions of variance for all principal components are:

[  9.99188643e-01,   5.22457527e-04,   2.84454714e-04,   4.18757463e-06,   2.57440701e-07,   3.72234266e-31,   6.94589339e-32,   6.91337479e-33]
Of the first three dimensions, we keep only the first and third one, although the variance of the second dimension is more than twice than that of the third. In fact, we found no appreciable dependence of the fit function on the second dimension (the derivative is ~ 0).

Assuming linearity in colors, the model is:
\begin{subequations}
\begin{equation}
y_{\rm nonlin2}=(b x_{\rm H2O}+1)^{I}-1 - s x_{\rm H2O} +a_{17}x_{\rm H2O}^2+a_{18}x_{\rm H2O}^3  
\end{equation}
\begin{equation}
b = a_0+a_1Y_0+a_2Y_2+a_3x_{\rm O3}+a_4x_{\rm tau}+a_5x_{\rm index}  
\end{equation}
\begin{equation}
I=a_6+a_7Y_0+a_8Y_2+a_9x_{\rm O3}+a_{\rm 10}x_{\rm tau}+a_{\rm 11}x_{\rm index}  
\end{equation}
\begin{equation}
s=a_{\rm 12} + a_{\rm 13}Y_0 + a_{\rm 14}x_{\rm O3} +  a{\rm 15}x_{\rm tau} + a_{\rm 16}x_{\rm index}
\end{equation}
\end{subequations}
where $Y_0$: the first PCA component of colors, $Y_2$: the third PCA component; $x_{\rm H2O}$: the $H_2O$ gas norm, $x_{\rm O3}$: the $O_3$ gas norm. The 'norm' unit is the default value of gas components used in Phosim, can be considered as a typical value, as mentioned before.
$x_{\rm tau}$ and $x_{\rm index}$: the coefficients aerotau and aeroindex in the aerosol optical depth formula: $AOD= aerotau(\frac{\lambda}{0.5\mu m})^{aeroindex}$

\paragraph{}
Of the 1333 wDs stars in the library, we focused on 526 stars selected from : 173 Bergeron 4750-6600K (indices: 177-350) and 352 Bergeron He 6600-30000 
(indices: 980-1332)
\paragraph{}
As a result, the coefficients for the y band are:

a = [5.77154789e+00,  -3.20399873e-02,   1.08998053e+00,   1.55738561e-03,   2.33856994e-01,   4.62531644e-03,  2.00091306e-02,   

2.10643367e-04,  -3.73130614e-03,  -2.56835242e-06,  -7.84909899e-04,  -4.69265058e-06,  -4.92006787e-02,  -5.23973422e-05,  

-1.48024115e-06,  -8.21841177e-06,  -7.81254315e-06,  -6.45271883e-03,  4.99160413e-04 ]

\paragraph{}
Overall the fit is fine for 526 stars: 4750-30000K, but poorer for cold stars: 2100-4750K and those with heavy absorption and/or emission lines that distort the originial curve shape:


\clearpage
\subsection{kurucz}
Results::
This database contains 4885 main sequence stars, we will still use Y0 and Y2 PCA components of colors. These spectra are not as smooth as the white dwarf stars we have chosen, and includes a lot absorption/emission lines and irregularities, butthe general shape is not distorted.

All the procedures are the same as before,

Of 8 colors: u-g, g-r, g-i, g-z, g-y, r-i, r-z, r-y, the intrinsic variances are of order 1: [0.42, 0.27, 0.40, 0.47, 0.54, 0.13, 0.21, 0.27]

The weight coefficients (descaled from the output weight matrix) for the first three eigenvectors  are:


W1:  [-0.7113498 , -0.86384771, -0.86451104, -0.86457053, -0.86510562,  -0.85761806, -0.85729617, -0.86145019]


W2:  [-3.4862384 , -0.05618584,  0.23751923,  0.31716404,  0.23846916,  0.82461389,  0.79499482,  0.52686915]


W3:  [ 0.43866125, -1.53329992, -0.90349735, -0.35336027, -0.20198974,  0.36887129,  1.16964883,  1.10829466]


The fractions of variance for all principal components are:

 9.51724529e-01,   4.66151055e-02,   1.50540159e-03,   1.20376340e-04,   3.45879943e-05,   1.37088008e-30,   9.08599591e-32,   5.07547727e-33

Y2 is 30 times smaller than Y1 though, we will still choose Y0 and Y2, and inclusion of Y1 does not make a difference.

The same model is used:

\paragraph{}
All the 4885 kurucz stars are used: 3830 - 11100 K (indices: 0 - 4884)
\paragraph{}
As a result, the coefficients of the model for the y band are:

a = [ 6.04269446e+00,  -3.18151272e-02,   7.34639205e-02,   3.33944920e-03,   1.33328943e-01,   7.43119284e-03,   1.84777039e-02,
      1.20141245e-04,  -2.36283230e-05,   -5.96763619e-06,  -6.02706874e-04,  -1.11935059e-05,  -5.05515089e-02,  -4.51061880e-05,  
      -2.49369440e-06,   1.26983866e-04,  -6.59163051e-06,  -6.95623194e-03,   5.56817453e-04 ]

\paragraph{}
Overall the fit is fine,  while the cold stars (g-r less than -0.4) have bigger residuals.


\clearpage
\subsection{Residuals}

To check the result, the model curves are compared with the data, for five kurucz stars: 3830K, 5430K, 5950K, and 7310K:
\paragraph{}
\graphicspath{{/home/bruno/users/weh40/Pictures/5figs/}}
\begin{figure}[h!]
\begin{center}
\includegraphics[width=1.\textwidth,trim=5mm 1.5mm 4mm 1.5mm,clip=true,angle=0]{curve_fit_checky_kurucz}
\caption{Check of curve fitting, y band, kurucz stars}
\end{center}
\end{figure}
\FloatBarrier

\clearpage
The residuals of kurucz stars are plotted with PWV=13.86 mm, along the g-r axis. Each color marks a different set of other components (aerosol, O3, O2). The scattering is within 0.001 mag.

\graphicspath{{/home/bruno/users/weh40/Pictures/5figs/}}
\begin{figure}[h!]
\begin{center}
\includegraphics[width=1.\textwidth,trim=1mm 1.5mm 1mm 1.5mm,clip=true,angle=0]{{diserrory1.0}.png}
\caption{fit residuals (in mag.) vs g-r, with PWV=13.86 mm, for y band}
\end{center}
\end{figure}
\FloatBarrier



\clearpage
In summary, the standard deviation and 90 percentile of residuals for both the selected wDs (black) and kurucz (red) stars are plotted against H2O PWV, with wDs std within 0.00013 mag, 90 percentile within0.00025 mag, and those of kurucz within 0.00023 and 0.00035 mag, respectively. This is reasonable, since kurucz stars have a lot of emission/absorption lines. 
\graphicspath{{/home/bruno/users/weh40/Pictures/5figs/}}
\begin{figure}[h!]
\begin{center}
\includegraphics[width=1.\textwidth,trim=1mm 1.5mm 1mm 1.5mm,clip=true,angle=0]{{dise_deviation_sumy}}
\caption{Standard Deviation of fit residuals}
\end{center}
\end{figure}
\FloatBarrier

\clearpage
\subsection{Prediction}


If we know all the other conditions, such as parameters of aerosol, ozone, oxygen, and colors, we can predict the H2O content based on the shift in y band magnitude. The python function we used is scipy.optimize.fsolve.
The pattern of the prediction residuals is similar to that of fit residuals.

\graphicspath{{/home/bruno/users/weh40/Pictures/5figs/}}
\begin{figure}[h!]
\begin{center}
\includegraphics[width=1.\textwidth,trim=1mm 1.5mm 1mm 1.5 mm,clip=true,angle=0]{{cgas_reverse_errory1.0_0}.png}
\caption{PWV prediction residuals vs g-r, with true PWV=13.86 mm, for y band. The black solid line indicate the wDs stars, the red dashed lines indicate the kurucz stars. T}
\end{center}
\end{figure}
\FloatBarrier

 To mimic the realistic situation, we can add noise to the band magnitude. To this end, we will add two uniform distribution noisees of mag 0.002 (S/N = 100 for PWV =  70mm, for which the bands mag. shift $\sim$ 0.21), and 0.01 mag (S/N = 20 for PWV = 70mm).

\paragraph{}
In summary, the prediction residuals in PWV H2O for both wDs and kurucz with and without noises are plotted in the following figure.
\graphicspath{{/home/bruno/users/weh40/Pictures/5figs/}}
\begin{figure}[h!]
\begin{center}
\includegraphics[width=1.\textwidth,trim=1mm 1.5mm 1mm 1.5mm,clip=true,angle=0]{{cgas_deviation_sumy_90}}
\caption{Deviation of prediction residuals, the black color indicates the wDs stars, the red color indicates the kurucz stars. The solid lines are for prediction residuals without noises, dashed lines for noise=0.002 mag, and dotted lines for noise=0.01 mag, and averaged value with 100 stars as well. The standard deviation (std) and 90 percentile (90p) lines are drawn for all cases with and without noises.}
\end{center}
\end{figure}
\FloatBarrier



\paragraph{}
From this figure, it can seen that all kinds of deviation increase with PWV, where the standard deviations of the prediction residuals for both types of stars are around 0.1 mm (PWV = 70 mm), 90 percent residuals around 0.1 - 0.2 mm.

\paragraph{}
With 0.002 mag noise (marked with triangles), there is virtually no distinction between wDs and kurucz stars, and all the prediction residuals are within 0.5 mm for PWV less than 13.86 mm. Their 90 percentiles become worse than 0.5 mm afterwards and reach 1 mm at PWV=70 mm. Averaged with 100 stars (marked with hexgons), the achieved accuracy (estimated based on 90 percentiles) will drop below 0.1 mm for PWV less than 13.86 mm, and become worth than 0.1 mm and reach 0.2 mm for wDs and 0.3 mm for kurucz stars, respectively. 

\paragraph{}
With 0.01 mag noise, the residuals are worse and it is the same for both types of stars. All the 90 percentile residuals are within 2 mm for PWV less than 13.86 mm, and grow worse afterwards and reach 5 mm at PWV=70mm. Averaged with 100 stars (marked with diamonds), the achieved accuracy (estimated with 90 percentiles) will drop below 0.2 mm for PWV less than 13.86 mm, and become worth than 0.2 mm and reach 0.6 mm for wDs and 0.7 mm for kurucz stars, respectively. 

\paragraph{}
(the estimate for what accuracy can be achieved with 100 stars is based on the square root law and considering bias of 90 percentile.)


\clearpage
\begin{thebibliography}{widestlabel}
\bibitem{manual}
The Photon Simulator (Phosim), John R. peterson and the Phosim Group, August 2013
\bibitem{atmoswater}
Ralph FK \& Stephen RS, Nature Vol 358 1992 "Seasonal and interannual variations in atmospheric oxygen and implications for the global carbon cycle"
\end{thebibliography}

\paragraph{}
\textbf{Appendices: } codes for integration (with annotarion)
%\end{document}